%%%%%%%%%%%%%%%%%%%%%%%%%%%%%%%%%%%%%%%%%%%%%%%%%%%%%%%%%%%%%%%%%%%%%%%%%%%%%%%%
%2345678901234567890123456789012345678901234567890123456789012345678901234567890
%        1         2         3         4         5         6         7         8

\documentclass[letterpaper, 10 pt, conference]{ieeeconf}  % Comment this line out
                                                          % if you need a4paper
%\documentclass[a4paper, 10pt, conference]{ieeeconf}      % Use this line for a4
                                                          % paper

\IEEEoverridecommandlockouts                              % This command is only
                                                          % needed if you want to
                                                          % use the \thanks command
\overrideIEEEmargins
% See the \addtolength command later in the file to balance the column lengths
% on the last page of the document

\usepackage[utf8]{inputenc}
\usepackage[T1]{fontenc}
\usepackage[english]{babel}
\usepackage{graphicx}
\usepackage{placeins}

% The following packages can be found on http:\\www.ctan.org
% \usepackage{graphics} % for pdf, bitmapped graphics files
%\usepackage{epsfig} % for postscript graphics files
%\usepackage{mathptmx} % assumes new font selection scheme installed
%\usepackage{mathptmx} % assumes new font selection scheme installed
%\usepackage{amsmath} % assumes amsmath package installed
%\usepackage{amssymb}  % assumes amsmath package installed

\title{\LARGE \bf
Design of a Public Protection and Disaster Relief Network for São Miguel Island}

\author{ \parbox{3 in}{\centering Daniel Rebelo\\
        % \thanks{*Use the $\backslash$thanks command to put information here}\\
        Instituto Superior Técnico / Instituto de Telecomunicações\\
        University of Lisbon\\
        Lisbon, Portugal\\
        {\tt\small danielrebelo@tecnico.ulisboa.pt}}
        \hspace*{ 0.5 in}
        \parbox{3 in}{ \centering António L. Topa\\
        %\thanks{**The footnote marks may be inserted manually}\\
        Instituto Superior Técnico / Instituto de Telecomunicações\\
        University of Lisbon\\
        Lisbon, Portugal\\
        {\tt\small antonio.topa@lx.it.pt}}
}

% \author{Daniel Rebelo% <-this % stops a space
% \thanks{*This work was not supported by any organization}% <-this % stops a space
% \thanks{$^{1}$H. Kwakernaak is with Faculty of Electrical Engineering, Mathematics and Computer Science,
%         University of Twente, 7500 AE Enschede, The Netherlands
%         {\tt\small h.kwakernaak at papercept.net}}%
% \thanks{$^{2}$P. Misra is with the Department of Electrical Engineering, Wright State University,
%         Dayton, OH 45435, USA
%         {\tt\small p.misra at ieee.org}}%
% }


\begin{document}



\maketitle
\thispagestyle{empty}
\pagestyle{empty}


%%%%%%%%%%%%%%%%%%%%%%%%%%%%%%%%%%%%%%%%%%%%%%%%%%%%%%%%%%%%%%%%%%%%%%%%%%%%%%%%
\begin{abstract}
This Dissertation aims at a project of a TETRA emergency and public safety network for São Miguel Island. In the Azores, there are numerous situations of risk to people and goods every year. The damage caused by these events is partly aggravated by human occupation in areas of high sensitivity and fragility. These factors, as well as recent events in Pedrogão Grande, aroused interest in developing a project on this topic. The objective of this Dissertation is to reach mobile network coverage values greater than or equal to 95\% and reach at least 98\% of the population. For this, the use of TETRA technology is used.\par\noindent
During this Dissertation are presented the two systems implemented in the Autonomous Region of the Azores. These systems are RITERAA and SIRESP.\par\noindent
During implementation of a mobile network, for coverage and interference reasons, it is necessary to estimate the signal transmitted by the base station. Signal estimation involves the calculation of an average value and the oscillations around this value. To estimate the signal, the Okumura-Hata model and the Walfisch-Ikegami model were used.\par\noindent
The fundamental concepts of mobile networks were discussed, followed by the project implementation. Finally, simulations were performed from the values  obtained in the project sizing. It was concluded that it is possible to exceed the objectives initially set since the coverage exceeds 96\% of the study area and reaches at least 99\% of the population.
\end{abstract}


%%%%%%%%%%%%%%%%%%%%%%%%%%%%%%%%%%%%%%%%%%%%%%%%%%%%%%%%%%%%%%%%%%%%%%%%%%%%%%%%
\section{INTRODUCTION}
\noindent In the Azores are known numerous situations of risk to people and goods, the consequences of these occurrences are magnified by human occupation in areas of high sensitivity and fragility.
Every year many natural disasters victimize people and cause harm around the world. In the case of the Azores archipelago, it is the floods, slope movements, earthquakes and natural hazards that register the highest number of occurrences, causing high material damage and, in extreme situations, fatalities. \par \noindent
The small size of the islands, the volcanic nature that determines the geomorphology and geology, are factors that should be considered. In situations of heavy rainfall, devastating flooding may occur, especially in urban areas located near the streams. 
Due to the characteristics that these types of calamities usually present, it is difficult to make their prediction, creating the need for an effective emergency communication system. \par \noindent
Thus, the opportunity arose to carry out a project of an emergency and safety communications network designed explicitly for the São Miguel Island of the Eastern Group of the Azores.\par\noindent
These networks have certain factors that must be considered. In commercial networks, those factors are negligible. The interest in these networks was sparked after the recent tragedy in Pedrogão Grande, in which SIRESP played a key role. Recently, emergency and public safety networks had a crucial role in the passage of Hurricane Lorenzo through the Archipelago of the Azores.\par\noindent
This paper elaborates on a Public Protection and Disaster Relief Network for São Miguel Island. This network is developed through TETRA technology, being necessary to consider the particularities of these systems.\par\noindent
In addition to mobile coverage planning, this paper also establishes the transport network. The transport network interconnects all the base stations of this network, enabling communications with any user, the transport network will use microwave links to interconnect the base stations. \par\noindent
This network will have to overcome the difficulties provided by the island (large water surfaces on the island, extremely irregular orography).
\begin{figure}[ht]
    \centering
    \includegraphics[width=0.4\textwidth]{Lorenzo.jpg}
    \caption{Hurricane Lorenzo}
    \label{fig:lorenzo}
\end{figure}
This paper aims to implement TETRA mobile coverage on São Miguel Island. It is intended that the region of interest has a coverage higher than 95\% of the region, serving at least 98\% of the population. It is desirable to have a similar amount of base stations of the current system. Software simulations must validate these results. The microwave links must have redundancies in order to have a functional and viable system even if there is a problem in a link.
%%%%%%%%%%%%%%%%%%%%%%%%%%%%%%%%%
\section{State of The Art And Basic Concepts Of TETRA Systems}

\subsection{State of the Art}

Public Protection and Disaster Relief Networks execute a vital role in the protection and operationality of all countries. These networks are intended to ensure communications between differents agencies, thereby ensuring interaction between all responders in emergency response. Thus, there is the possibility of producing the necessary communications between these entities at any time of the emergency\par\noindent
Hence, with resources and access to this infrastructure, it is possible to plan, coordinate, and execute the emergency plan as needed.\par\noindent
Currently, several technologies allow the implementation of this type of networks, such as:
\begin{itemize}
    \item \textbf{DMR} - Digital Mobile Radio;
    \item \textbf{dPMR} - Digital Private Mobile Radio;
    \item \textbf{LTE/PPDR} - Long Term Evolution / Public Protection and Disaster Relief;
    \item \textbf{TETRA} - Terrestrial Trunked Radio.
\end{itemize}
\par\noindent
This PPDR network is implemented with TETRA technology. Therefore, this paper describes and explains the basics of this technology. %%%%%%%%%%%%%%%%%%%%%%%
\subsection{Terrestrial Trunked Radio}
\noindent TETRA technology was standardized by ETSI and is supplied by various companies, is designed for UHF (Ultra High Frequency) voice and data transmission. TETRA has, in recent years, been the leading technology for public and private mobile radio systems, especially for public safety networks. It has been designed to have a highly stable and operational system with services that are adjustable to the needs of users. The implementation of this network can be found in most European countries (Germany, Austria, Belgium, Croatia, Finland, Greece, the Netherlands, Hungary, Ireland, Luxembourg, Norway, Portugal, Romania, Sweden, and the United Kingdom, among others).
\par\noindent
TETRA technology is a digital transmission system that uses two access methods, FDMA and TDMA. The air interface of this system is divided into carriers with different frequencies (FDMA). Each of these carriers is divided into four time-slots (TDMA). Each time-slot is a communication channel, and this factor allows each carrier to have up to four separate communications occurring at the same time.\par\noindent
Each carrier has a bandwidth of $25kHz$, linking this with the fact that there are four channels per carrier, it is possible in $200kHz$ to have thirty-two communication channels. If this technology were only built on FDMA, there would only be eight channels available in a bandwidth of  $200kHZ$. As a result of the unification of the two access methods used by TETRA, it can be said that spectrum use is very efficient and much more efficient than other technologies.
\begin{figure}
    \centering
    \includegraphics[width=0.45\textwidth]{TETRA_spectrum.JPG}
    \caption{Tetra spectrum usage [RETIRADO DE]}
    \label{fig:TETRAspectrum}
\end{figure}
The European Radiocommunications Committee established  the following frequency bands that this system can be operated:
\begin{itemize}
    \item  \hspace{1.2mm}   [380-385] MHz paired with [390-395] MHz for emergency services;
    \item   \hspace{1.2mm}  [385-390] MHz paired with [395-400] MHz for public services;
    \item   \hspace{1.2mm}  [410-420] MHz paired with [420-430] MHz for civil use.
\end{itemize}
\par\noindent
In PPDR  networks, it is necessary that the system easily and quickly adapts to the needs of users. To meet this need, TETRA has mechanisms and configurations that maintain system availability, allowing greater communications flexibility even during emergencies.
\par\noindent This mode also There are two main modes of operation: DMO and TMO. DMO (Direct Mode Operation) provides direct communication between two handhelds without using a base station. It can be used when the local system capacity is fully occupied or when the handheld is out of the area of coverage.\par\noindent
Besides allowing direct communications between handhelds, the Direct Mode offers the possibility of communicating through a repeater or a gateway.
\begin{itemize}
    \item \textbf{Repeater} - If two handhelds are too far apart, and out of the coverage zone of the handheld, another equipment can be configured to serve as a repeater, and therefore enables the use of direct communications.
    \item \textbf{Gateway} - It is possible to configure a mobile station to operate as a gateway. This way, this station will connect users who are out of range of the base station to the network. For example, in areas where there is no signal coverage due to terrain orography, placing one of these stations within the boundary of the base station's coverage area may be sufficient to connect users who are out of range of the base station to the network. Therefore, with this configuration, it is possible to connect the users to the control and management center.
\end{itemize}
\par\noindent Direct Mode operation is illustrated in the next figure
\begin{figure}[h]
    \centering
    \includegraphics[width=0.45\textwidth]{DMO_PAPER.jpg}
    \caption{Direct Mode REF[]}
    \label{fig:DMO}
\end{figure}
\FloatBarrier
\subsection{SIRESP and RITERAA} 
Currently, in Portugal, the implemented PPDR network is SIRESP.  
This network was developed based on TETRA technology and was expected to cover the entire continental territory and the archipelagos.\par\noindent
It was planned that the implementation of this network would be performed in 7 distinct phases, thus guaranteeing the full functionality of this network throughout the Portuguese territory.
Currently, the SIRESP Network consists of a mobile radio coverage that ensures connection to mobile terminals, and in a fixed network.
The system has indeed reached the archipelago, but only for entities that are dependent on the state.\par\noindent
Thus, it appears that this resolution does not fit the objectives of the SIRESP network, which would be to manage all the agencies involved in security and emergency in the region.\par\noindent
The Azores Regional Civil Protection and Fire Service (SRPCBA) continued with its own communications system, but with a different operating frequency from that used in the SIRESP system. The developed system for the SRPCBA is RITERAA.
\begin{figure}
    \centering
    \includegraphics[width=0.35\textwidth]{SirespCobertura.png}
    \caption{S}
    \label{fig:SIRESP}
\end{figure}
\FloatBarrier
\noindent
The  Regional Government of the Azores did not accept the full implementation of the system in the archipelago. This decision was based on various factors. Those factors were financial, technical, geographical, and meteorological.\par\noindent
Another reason, provided by the responsible entities, was that the system designed to operate in the Azores paid no particular attention to the terrain orography neither to the natural geographical discontinuity of the archipelago. This factor could lead to a lack of coverage in some regions. Also, the implementation of  SIRESP would not be the most suitable for making connections between the islands of the archipelago.\par\noindent
RITERAA was developed in 2014. Instead of TETRA, this network uses DMR technology. RITERAA allows reaching coverage of $ 95 \% $ of the Azorean territory, reaching $ 98 \% $ of the population, thus allowing greater efficiency and reliability in communications. São Miguel Island has eleven base station. The localization of these base station are shown in the next figure. 
\begin{figure}[h]
    \centering
    \includegraphics[width=0.45\textwidth]{riteraaestacoesbase.jpg}
    \caption{Localization of the base stations in São Miguel Island}
    \label{fig:RITERAA}
\end{figure}
\FloatBarrier
%%%%%%%%%
\section{Fundamental Concepts}

Use this sample document as your LaTeX source file to create your document. Save this file as {\bf root.tex}. You have to make sure to use the cls file that came with this distribution. If you use a different style file, you cannot expect to get required margins. Note also that when you are creating your out PDF file, the source file is only part of the equation. \emph{Your \TeX\ $\rightarrow$ PDF filter determines the output file size. Even if you make all the specifications to output a letter file in the source - if you filter is set to produce A4, you will only get A4 output.}

It is impossible to account for all possible situation, one would encounter using \TeX. If you are using multiple \TeX\ files you must make sure that the ``MAIN`` source file is called root.tex - this is particularly important if your conference is using PaperPlaza's built in \TeX\ to PDF conversion tool.

\subsection{Headings, etc}

Text heads organize the topics on a relational, hierarchical basis. For example, the paper title is the primary text head because all subsequent material relates and elaborates on this one topic. If there are two or more sub-topics, the next level head (uppercase Roman numerals) should be used and, conversely, if there are not at least two sub-topics, then no subheads should be introduced. Styles named ``Heading 1'', ``Heading 2'', ``Heading 3'', and ``Heading 4'' are prescribed.

\subsection{Figures and Tables}

Positioning Figures and Tables: Place figures and tables at the top and bottom of columns. Avoid placing them in the middle of columns. Large figures and tables may span across both columns. Figure captions should be below the figures; table heads should appear above the tables. Insert figures and tables after they are cited in the text. Use the abbreviation ``Fig. 1'', even at the beginning of a sentence.

\begin{table}[h]
\caption{An Example of a Table}
\label{table_example}
\begin{center}
\begin{tabular}{|c||c|}
\hline
One & Two\\
\hline
Three & Four\\
\hline
\end{tabular}
\end{center}
\end{table}


   \begin{figure}[thpb]
      \centering
      \framebox{\parbox{3in}{We suggest that you use a text box to insert a graphic (which is ideally a 300 dpi TIFF or EPS file, with all fonts embedded) because, in an document, this method is somewhat more stable than directly inserting a picture.
}}
      %\includegraphics[scale=1.0]{figurefile}
      \caption{Inductance of oscillation winding on amorphous
       magnetic core versus DC bias magnetic field}
      \label{figurelabel}
   \end{figure}
   

Figure Labels: Use 8 point Times New Roman for Figure labels. Use words rather than symbols or abbreviations when writing Figure axis labels to avoid confusing the reader. As an example, write the quantity ``Magnetization'', or ``Magnetization, M'', not just ``M''. If including units in the label, present them within parentheses. Do not label axes only with units. In the example, write ``Magnetization (A/m)'' or ``Magnetization {A[m(1)]}'', not just ``A/m''. Do not label axes with a ratio of quantities and units. For example, write ``Temperature (K)'', not ``Temperature/K.''

\section{CONCLUSIONS}

A conclusion section is not required. Although a conclusion may review the main points of the paper, do not replicate the abstract as the conclusion. A conclusion might elaborate on the importance of the work or suggest applications and extensions. 

\addtolength{\textheight}{-12cm}   % This command serves to balance the column lengths
                                  % on the last page of the document manually. It shortens
                                  % the textheight of the last page by a suitable amount.
                                  % This command does not take effect until the next page
                                  % so it should come on the page before the last. Make
                                  % sure that you do not shorten the textheight too much.

%%%%%%%%%%%%%%%%%%%%%%%%%%%%%%%%%%%%%%%%%%%%%%%%%%%%%%%%%%%%%%%%%%%%%%%%%%%%%%%%



%%%%%%%%%%%%%%%%%%%%%%%%%%%%%%%%%%%%%%%%%%%%%%%%%%%%%%%%%%%%%%%%%%%%%%%%%%%%%%%%



%%%%%%%%%%%%%%%%%%%%%%%%%%%%%%%%%%%%%%%%%%%%%%%%%%%%%%%%%%%%%%%%%%%%%%%%%%%%%%%%
\section*{APPENDIX}

Appendixes should appear before the acknowledgment.

\section*{ACKNOWLEDGMENT}

The preferred spelling of the word ``acknowledgment'' in America is without an ``e'' after the ``g''. Avoid the stilted expression, ``One of us (R. B. G.) thanks . . .''  Instead, try ``R. B. G. thanks''. Put sponsor acknowledgments in the unnumbered footnote on the first page.



%%%%%%%%%%%%%%%%%%%%%%%%%%%%%%%%%%%%%%%%%%%%%%%%%%%%%%%%%%%%%%%%%%%%%%%%%%%%%%%%

References are important to the reader; therefore, each citation must be complete and correct. If at all possible, references should be commonly available publications.



\begin{thebibliography}{99}

\bibitem{c1} G. O. Young, ``Synthetic structure of industrial plastics (Book style with paper title and editor),'' 	in Plastics, 2nd ed. vol. 3, J. Peters, Ed.  New York: McGraw-Hill, 1964, pp. 15--64.
\bibitem{c2} W.-K. Chen, Linear Networks and Systems (Book style).	Belmont, CA: Wadsworth, 1993, pp. 123--135.
\bibitem{c3} H. Poor, An Introduction to Signal Detection and Estimation.   New York: Springer-Verlag, 1985, ch. 4.
\bibitem{c4} B. Smith, ``An approach to graphs of linear forms (Unpublished work style),'' unpublished.
\bibitem{c5} E. H. Miller, ``A note on reflector arrays (Periodical styleÑAccepted for publication),'' IEEE Trans. Antennas Propagat., to be publised.
\bibitem{c6} J. Wang, ``Fundamentals of erbium-doped fiber amplifiers arrays (Periodical styleÑSubmitted for publication),'' IEEE J. Quantum Electron., submitted for publication.
\bibitem{c7} C. J. Kaufman, Rocky Mountain Research Lab., Boulder, CO, private communication, May 1995.
\bibitem{c8} Y. Yorozu, M. Hirano, K. Oka, and Y. Tagawa, ``Electron spectroscopy studies on magneto-optical media and plastic substrate interfaces(Translation Journals style),'' IEEE Transl. J. Magn.Jpn., vol. 2, Aug. 1987, pp. 740--741 [Dig. 9th Annu. Conf. Magnetics Japan, 1982, p. 301].
\bibitem{c9} M. Young, The Techincal Writers Handbook.  Mill Valley, CA: University Science, 1989.
\bibitem{c10} J. U. Duncombe, ``Infrared navigationÑPart I: An assessment of feasibility (Periodical style),'' IEEE Trans. Electron Devices, vol. ED-11, pp. 34--39, Jan. 1959.
\bibitem{c11} S. Chen, B. Mulgrew, and P. M. Grant, ``A clustering technique for digital communications channel equalization using radial basis function networks,'' IEEE Trans. Neural Networks, vol. 4, pp. 570--578, July 1993.
\bibitem{c12} R. W. Lucky, ``Automatic equalization for digital communication,'' Bell Syst. Tech. J., vol. 44, no. 4, pp. 547--588, Apr. 1965.
\bibitem{c13} S. P. Bingulac, ``On the compatibility of adaptive controllers (Published Conference Proceedings style),'' in Proc. 4th Annu. Allerton Conf. Circuits and Systems Theory, New York, 1994, pp. 8--16.
\bibitem{c14} G. R. Faulhaber, ``Design of service systems with priority reservation,'' in Conf. Rec. 1995 IEEE Int. Conf. Communications, pp. 3--8.
\bibitem{c15} W. D. Doyle, ``Magnetization reversal in films with biaxial anisotropy,'' in 1987 Proc. INTERMAG Conf., pp. 2.2-1--2.2-6.
\bibitem{c16} G. W. Juette and L. E. Zeffanella, ``Radio noise currents n short sections on bundle conductors (Presented Conference Paper style),'' presented at the IEEE Summer power Meeting, Dallas, TX, June 22--27, 1990, Paper 90 SM 690-0 PWRS.
\bibitem{c17} J. G. Kreifeldt, ``An analysis of surface-detected EMG as an amplitude-modulated noise,'' presented at the 1989 Int. Conf. Medicine and Biological Engineering, Chicago, IL.
\bibitem{c18} J. Williams, ``Narrow-band analyzer (Thesis or Dissertation style),'' Ph.D. dissertation, Dept. Elect. Eng., Harvard Univ., Cambridge, MA, 1993. 
\bibitem{c19} N. Kawasaki, ``Parametric study of thermal and chemical nonequilibrium nozzle flow,'' M.S. thesis, Dept. Electron. Eng., Osaka Univ., Osaka, Japan, 1993.
\bibitem{c20} J. P. Wilkinson, ``Nonlinear resonant circuit devices (Patent style),'' U.S. Patent 3 624 12, July 16, 1990. 






\end{thebibliography}




\end{document}
